\section{Carga y limpieza de datos}

Para iniciar con la carga y limpieza de datos, primero se creo el módulo \comillas{paths} en donde se hará registro de todas las rutas necesarias para realizar la tarea. Además, se generan las rutas \codigo{PATH\_DATA} y \codigo{PATH\_RAW}, que serán a las carpetas \comillas{data} y \comillas{raw} respectivamente. Luego se utilizará el diccionario \codigo{dict\_path\_raw}, para que al entregarle la llave \comillas{wNN}, indicará la ruta de la carpeta \comillas{wNN} en \comillas{raw}. Sumado a este diccionario se crean otros dos, \codigo{dict\_csv\_mc\_a} y \codigo{dict\_csv\_mc\_f}, con el fin de que entreguen la ruta para los archivos .csv, \codigo{metrocuadrado\_all\_wNN} y \codigo{metrocuadrado\_furnished\_wNN} respectivamente, al darles de llave \comillas{wNN}.

Después, con el fin generar \comillas{furnished} y de eliminar los datos duplicados, se procede a crear un único DataFrame que contenga la información de los datasets proporcionados. Primero se crean dos listas de DataFrames, uno para \comillas{all} y el otro para \comillas{furnished}, para luego concatenar los DataFrames en dos DataFrames únicos, nuevamente uno para \comillas{all} y otro para \comillas{furnished}. Así, se crea una nueva columna \comillas{furnished}, que informa si el DataFrame proviene de \comillas{furnished}, que será indicado con un 1, o si viene de \comillas{all}, indicado con un 0. Posteriormente se unen los dos DataFrames anteriores en uno único, añadiendo la columna extra creada, para finalmente eliminar los duplicados con la función \codigo{drop\_duplicates} de pandas.

Para limpiar a columnas \codigo{price}, \codigo{n\_rooms}, \codigo{n\_bath} y \codigo{surface}, se procede a crear el diccionario \codigo{valores\_nulos} que tendrá el nombre de las columnas del dataframe como llave, donde cada valor será lo que se define como nulo para esa columna. Aparte se cambian los tipos de datos de las columnas precio y área del inmueble, con el fin de usar tipos int y float respectivamente. Luego, para crear nuevas columnas con el tipo de inmueble, el tipo de oferta y el barrio, primero se crea una copia de las columnas \codigo{property\_type}, \codigo{rent\_type} y \codigo{location} con palabras en minúsculas, para unificar su revisión. Se crean listas con la información respectiva al recorrer variantes de estas últimas columnas, las cuales que indican la información necesaria, para finalmente agregarlas como columnas.

Nuevamente se pretende agregar columnas, \codigo{price\_per\_m2} y \codigo{garajes}, con el fin de poder analizar mejor la información. Para ello se desarrollarán funciones que procesen las columnas \codigo{url} y en base a esas se crearán nuevas columnas a partir de las columnas que se mapeen, donde finalmente se agregarán al dataset.

Para crear las 8 clasificaciones que se pide hacer, se utiliza el comando query, además de crearse las columnas que representan dicha clasificación. En este proceso se seleccionaron aquellas filas que cumplieran con alguna de las 8 clasificaciones, por lo que de haber alguna que no se podía clasificar, fue eliminada implícitamente.

Luego se procede a realizar una columna con el código UPZ de cada inmueble, utilizando la columna \comillas{barrio} y el archivo \codigo{data/asignacion\_upz/barrio-upz-asignacion.csv}. A partir de esto se calcula que existen 150 barrios sin código UPZ, lo que se traduce en este caso en que aproximadamente el $91\%$ de los casos tiene la información de UPZ, que es suficiente para seleccionarlos y poder trabajar con ellos.

Para finalizar la carga de datos, se fusionaron los archivos pedidos al \codigo{DataFrame}. Se carga datos sobre el índice de inseguridad, el porcentaje de áreas verdes y las estadísticas de la población, revisando que no hayan casos en que haya más de una o ninguna información para cada observación. Al realizar esto, solo en el caso de índice de seguridad se dio una observación distinta que se procedió a eliminar, para evitar casos raros. Finalmente se crea una columna para la densidad de población, utilizando el valor \codigo{UPlArea} y el número de personas.